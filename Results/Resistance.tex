\subsection{Resistance}
\label{sec:results_resistance}
These are the results for our resistance measurement feature. \\ Range from 10$\Omega$ - 3M$\Omega$.
\begin{figure}[h]
    \centering
    \begin{tikzpicture}
    \begin{axis}[legend style={at={(0.95, 0.09)},anchor=south east}, width=\linewidth, height=6cm,
        title = {Resistance Measurement},
        xlabel = {Target [$\Omega$]},
        ylabel = {Measured [$\Omega$]}]
        \addplot[blue, mark=*] table {Results/ResData/resDataCal.dat};
        \addlegendentry{Calibrated}
        \addplot[red, mark=*] table {Results/ResData/resDataMeter.dat};
        \addlegendentry{Ohmmeter}
    \end{axis}
    \end{tikzpicture}
    \caption{Comparing the results between our ohmmeter and the calibrated one}
    \label{fig:resGraph}
\end{figure}

\noindent The graph in Figure \ref{fig:resGraph} shows two plots. The calibrated one refers to a Keysight 34461A Digital Multimeter that was calibrated in 2023 by Trescal, while the other refers to our own ohmmeter in the multimeter. The same leads were used throughout.

\begin{figure}[h]
    \centering
    \begin{tikzpicture}
    \begin{axis}[legend style={at={(0.95, 0.09)},anchor=south east}, width=\linewidth, height=4cm,
        title = {Percentage deviation, resistance},
        xlabel = {Target [$\Omega$]},
        ylabel = {Percentage [\%]}]
        \addplot[red, mark=*] table {Results/ResData/resDiff.dat};
    \end{axis}
    \end{tikzpicture}
    \caption{The percentage difference from calibrated value and measured}
    \label{fig:resGraphDiff}
\end{figure}

\noindent Seen in Figure \ref{fig:resGraphDiff}, the deviation of the measured values increases greatly between our set reference resistors.
\FloatBarrier
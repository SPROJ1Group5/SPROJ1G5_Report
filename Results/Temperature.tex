\subsection{Temperature}
\label{sec:results_temperature}
These are the results of our temperature measurement feature. Range from 5$^\circ$C - 40$^\circ$C. This was achieved with the Vötsch VT 7004 programmable temperature chamber and the calibrated thermometer Fluke t3000 FC as reference.

\begin{figure}[hbt!]
    \centering

    \begin{tikzpicture}

    \begin{axis}[legend style={at={(0.95, 0.09)},anchor=south east}, width=\linewidth, height=6cm,
        title = {Temperature},
        xlabel = {Target [$^\circ$C]},
        ylabel = {Measured [$^\circ$C]}]
        \addplot[blue, mark=*] table {Results/TempData/tempDataCal.dat};
        \addlegendentry{Calibrated}
        \addplot[red, mark=*] table {Results/TempData/tempDataTM.dat};
        \addlegendentry{Thermometer}
    \end{axis}

    \end{tikzpicture}

    \caption{Comparing the results between our thermometer and the calibrated one}
    \label{fig:tempGraph}
    
\end{figure}

\noindent The sensor achieves a pretty linear performance, seen in Figure \ref{fig:tempGraph} - almost on par with the calibrated thermometer. Some offsets could be coded to increase accuracy.

\begin{figure}[h]
    \centering

    \begin{tikzpicture}

    \begin{axis}[legend style={at={(0.95, 0.8)},anchor=south east}, width=\linewidth, height=4cm,
        title = {Percentage deviation, temperature},
        xlabel = {Target [$^\circ$C]},
        ylabel = {Percentage [\%]}]
        \addplot[red, mark=*] table {Results/TempData/tempDataDiff.dat};
    \end{axis}

    \end{tikzpicture}

    \caption{The percentage difference from calibrated value and measured}
    \label{fig:tempGraphDiff}
    
\end{figure}

\noindent Figure \ref{fig:tempGraphDiff} shows a percentage swinging between 4.7\% and 15.3\%. Since the LM35 is attached to the PCB, thus having a larger mass to heat up, and the thermometer was hanging in the air, it was necessary to wait 15 minutes per 5-degree interval. Hence, this can create an offset between ambient and PCB. See Figure \ref{fig:TM1} and \ref{fig:TM2} for test setup.